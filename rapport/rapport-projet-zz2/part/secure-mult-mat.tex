%%%%%%%%%%%%%%%%%%%%%%%%%%%%%%%%%%%%%%%%%%%%%%%%%%%%%%%%%%%%%%%%%%%%%%%

%%%\section{Multiplication sécurisée de matrices via MapReduce}

%%%%%%%%%%%%%%%%%%%%%%%%%%%%%%%%%%%%%%%%%%%%%%%%%%%%%%%%%%%%%%%%%%%%%%%%


L'objectif étant la multiplication de matrices de manière sécurisée via MapReduce, 
comment peut-on lier la multiplication de matrices via MapReduce et le cryptosystème de Paillier?
C'est à cette question que nos tuteurs ont répondu dans leurs travaux de recherches \cite{publi-tuteurs} et que l'on va
présenter dans cette section.\par
\vspace{1\baselineskip}
Pour multiplier les matrices tout en gardant la confidentialité des données, avec MapReduce,
2 approches sont possibles.
\begin{itemize}
\item une approche privée et sécurisée (SP);
\item une approche privée et sécurisée, résistante aux chocs entre le map et le reduce (CRSP).
\end{itemize}
\littlesectionspace
Ces approches respectent les propriétés suivantes:
\begin{itemize}
\item l'utilisateur ne peut apprendre aucune information sur les matrices initiales M et N;
\item aucune machine s'occupant des données de la matrice M, ne peut apprendre des informations 
sur la matrice N;
\item aucune machine s'occupant des données de la matrice N, ne peut apprendre des informations 
sur la matrice M;  
\item aucune machine s'occupant du résultat de la multiplication de matrice $P= MN$ , ne peut 
apprendre des informations sur les matrices M,N et P.
\end{itemize}
Dans le cadre de notre projet, nous nous sommes uniquement intéressés à l'approche sécurisée et privée.
\vspace{1\baselineskip}

\subsection{Approche sécurisée et privée MapReduce Two Steps}

En plus d'utiliser le cryptosystème de Paillier, on ajoute un masque aléatoire pour assurer la confidentialité des éléments des matrices.
En effet, tous les éléments $m_{ij}$ de la matrice M sont cryptés via le cryptosystème de Paillier avec la clé public de l'utilisateur $pk$. 
De plus, ils masquent les éléments de la matrice N $n_{ij}$  en  leur ajoutant un nombre aléatoire  $\tau_{jk} \in {Z}_{n}$, 
également crypté mais avec une autre clé $pk_{2}$.
L'algorithme~\ref{algo_map1_SP} est associé à cette fonction.
\begin{algorithm}[H]
  \caption{Map SP round 1}
  \label{algo_map1_SP}
  \begin{algorithmic}
    \Require{matrices M et N}
    \Ensure{Liste de paires (clé, valeur)}
    \ForAll {$ m_{ij}$}
    \State Créer la paire $(j,[M,i,\epsilon_{pk}(m_{ij})])$
    \EndFor
    \ForAll {$ n_{jk}$}
    \State Créer la paire (j,[N,k, $n_{jk}$ + $\tau_{jk}$, $\epsilon_{pk_{2}}(\tau _{jk})$])
    \EndFor
  \end{algorithmic}
\end{algorithm}

Dans l'algorithme~\ref{algo_Reduce1_SP}, qui correspond à la première fonction reduce version sécurisée, il y'a aussi de nombreux changements.
Elle regroupe toujours les éléments qui ont la même clé $(i,k)$ , mais elle crée, cette fois-ci, comme valeur le tuple: ($\epsilon_{pk}(m_{ij})^{n_{jk} + \tau_{jk}}$ , $ \epsilon_{pk}(m_{ij})$ , $\epsilon_{pk_{2}}(\tau _{jk}$)
\littlesectionspace
\begin{algorithm}[H]
  \caption{Reduce SP round 1}
  \label{algo_Reduce1_SP}
  \begin{algorithmic}
    \Require{liste de paires \\  
    (j,[M,i,$\epsilon_{pk}(m_{ij})$]) ,(j,[N,k,$n_{jk} + \tau _{jk}, \epsilon_{pk_{2}}(\tau_{jk})$])}
	\\						  
    \Ensure{Liste  (cl\'{e}, liste de valeurs) \\ 
				   ((i,k), $\epsilon_{pk}(m_{ij})^{n_{jk} + \tau _{jk}}$,$ \epsilon_{pk}(m_{ij})$,$\epsilon_{pk_{2}}(\tau _{jk})$)}
    \\
    \ForAll { (j,[M,i,$\epsilon_{pk}(m_{ij})$]), (j,[N,k,$\epsilon_{pk}(n_{jk})$])}
    \State Cr\'{e}er  
			\[
				((i,k), \epsilon_{pk}(m_{ij})^{n_{jk} + \tau _{jk}}, \epsilon_{pk}(m_{ij}),\epsilon_{pk_{2}}(\tau _{jk}))
			\]
		
    \EndFor
  \end{algorithmic}
\end{algorithm}

Comme dans la version non sécurisée le deuxième map n'a pas grand intérêt vu qu'il renvoie la même sortie que la première fonction reduce.
La deuxième fonction reduce multiplie toutes les valeurs qui  porte la même clé $(i,k)$ et permet d'enlever les masques $\epsilon_{pk_{2}}(\tau_{jk})$, afin 
d'obtenir $\epsilon_{pk}( m_{ij} \times n_{kj})$, comme expliqué ci dessous et dans l'algorithme ci-dessous.
\littlesectionspace
\begin{algorithm}[H]
  \caption{Reduce SP round 2}
  \label{algo_Reduce2_SP}
  \begin{algorithmic} 
    \Require{Liste (cl\'{e}, liste de valeurs ) ((i,k), $\epsilon_{pk}(m_{ij})^{n_{jk} + \tau _{jk}}$,$ \epsilon_{pk}(m_{ij})$,$\epsilon_{pk_{r2}}(\tau _{jk})$)} 
    \Ensure{Liste de paires (cl\'{e}, valeur) ((i,k), $\epsilon_{pk}(\sum_{j}$ $m_{ij}$ $\times$ $n_{jk})$)}
    \ForAll { clé (i,k)}
    \State 
			\[
                   \prod_{j} \frac{ \epsilon_{pk}(m_{ij})^{n_{jk} + \tau _{jk}} }{ \epsilon_{pk}(m_{ij}^{\Delta_{sk_{r2}}(\epsilon_{pk_{r2}}(\tau_{jk}))}) }
             \]      
    \EndFor
  \end{algorithmic}
\end{algorithm}

\bigskip
On utilise les propriétés homomorphiques du cryptosystème de Paillier pour récupérer :
\begin{center}
$\epsilon_{pk}(\sum_{j} m_{ij} \times n_{jk})$
\end{center}
En effet:
%%
\[
	 \prod_{j} \frac{ \epsilon_{pk}(m_{ij})^{ n_{jk} + \tau_{jk} } }{ \epsilon_{pk}(m_{ij})^{\Delta_{sk_{r2}}(\epsilon_{pk_{r2}}(\tau_{jk}))}} =
	 %%
	  \prod_{j} \frac{ \epsilon_{pk}(m_{ij})^{n_{jk}} \times \epsilon_{pk}(m_{ij})^{\tau_{jk}} }{ \epsilon_{pk}(m_{ij})^{\tau_{jk}} } =
	 %%
	  \prod_{j} \epsilon_{pk}(m_{ij} \times  n_{jk}) =
	  %%
	  \epsilon_{pk}( \sum_{j} m_{ij} \times n_{jk}) 
\]

\subsection{Approche sécurisée et privée MapReduce One Step}

Dans la version One Step, il n'y a pas besoin de masquer les données provenant de la matrice N, vu qu'il n'y a qu'une seule fonction reduce.  
La fonction map chiffre seulement les données de la matrice M avec la clé publique de l'utilisateur. La fonction est décrite par l'algorithme~\ref{algo_map_OneStep_SP}

\begin{algorithm}[H]
  \label{algo_map_OneStep_SP}
  \begin{algorithmic}
    \Require{matrices M et N}
    \Ensure{Liste de paires (cl\'{e}, valeur)}
    \ForAll {$ m_{ij}$}
    \State \ForAll {k ,colonne de N}
    \State Cr\'{e}er la paire $((i,k),[M,j,\epsilon_{pk}(m_{ij})])$
    \EndFor
    \EndFor
    \ForAll {$ n_{jk}$}
	\State \ForAll {i ,ligne de M}
    \State Cr\'{e}er la paire $((i,k),[N,j,n_{jk}])$
    \EndFor
    \EndFor
  \end{algorithmic}
\caption{Map SP}
\end{algorithm}

En sortie de la fonction reduce nous retrouvons pour chaque clé $(i,k)$ , le produit  $m_{ij} \times n_{jk}$ chiffré par la clé publique de l'utilisateur $pk$.
\littlesectionspace 
\begin{algorithm}[H]
  \caption{Reduce SP}
  \label{algo_reduce_OneStep_SP}
  \begin{algorithmic}
    \Require{Liste de paires (cl\'{e}, valeur)}
    \Ensure{Liste de paires (cl\'{e}, valeur) ((i,k),  $\epsilon_{pk}$ ( $\sum_{k}$ $m_{ij}$ $\times$ $n_{jk}$))}
    \ForAll {$ (i,k)$}
    \State Cr\'{e}er une liste triée selon j de $[M,j,\epsilon_{pk}(m_{ij})]$
    \State Cr\'{e}er une liste triée selon j de $[N,j,n_{jk}]$
    \State Calculer
    \[
		\prod_{j} \epsilon_{pk}( m_{ij})^{n_{jk}}  
	\]
	\EndFor
  \end{algorithmic}
\end{algorithm}

\noindent On utilise, une nouvelle fois les propriétés homomorphiques pour retrouver:\par
\begin{center}
$\epsilon_{pk}$ ($\sum_{k}$ $m_{ij}$ $\times$ $n_{jk}$)
\end{center}

Rappel de la propriété homomorphique utilisée:
\[
	\prod_{j} \epsilon_{pk}( m_{ij})^{n_{jk}}  = 
	%%
	\prod_{j} \epsilon_{pk}( m_{ij} \times n_{jk} ) =
	%%
	 \epsilon_{pk}( \sum _{j}  m_{ij} \times n_{jk}  )
\]
